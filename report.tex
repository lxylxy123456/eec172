\documentclass[conference]{IEEEtran}
\usepackage{cite}
\usepackage{amsmath,amssymb,amsfonts}
\usepackage{algorithmic}
\usepackage{graphicx}
\usepackage{textcomp}
\usepackage{xcolor}
\usepackage{verbatim}
\usepackage{listings}
\def\BibTeX{{\rm B\kern-.05em{\sc i\kern-.025em b}\kern-.08em
    T\kern-.1667em\lower.7ex\hbox{E}\kern-.125emX}}
\begin{document}

\lstset{
	basicstyle=\ttfamily
}

\begin{comment}
Hi Class,

Please use the following format from IEEE for your final project report.
https://www.ieee.org/conferences/publishing/templates.html

The margin, font size are not important: rather your report should:
-have title
-two column
-have abstract, introduction, conclusion and other sections you want to add
-Have references
-Your report should have detailed information explaining the motivation,
 approach, components used and different communications or methodologies they
 have used.

-Demonstration video: 5 minutes is recommended. If you want to go longer you
 can.

-All the source code and detailed implementation details on how the experiment
 could be repeated again. This includes all the schematics and details about
 other services they might have used. For example, if someone has used api,
 where to get the api, what things need to be configured on the hosting side or
 on the user side.

-If using aws lambda, the functions used on lambda services and all the small
 details.

Houman
\end{comment}

\title{
 Implementing Optimized Brick Breaker in CC3200 with OLED and Accelerometer from
 Scratch
}

\author{\IEEEauthorblockN{Eric Li (Xiaoyi)}
\IEEEauthorblockA{\textit{University of California, Davis}\\
Davis, U.S.A. \\
ercli@ucdavis.edu}
}

\maketitle

\begin{abstract}
We can implement games with a CC3200 Launch Pad. However, a naive implementation
 will suffer from performance problems, such as limitation on SPI transfer
 speed. In this paper, I will go though how to implement and optimize a brick
 breaker game in CC3200. The game will use accelerometer, OLED, and the WiFi
 module. UART can be optionally used for debugging. 
\end{abstract}

\begin{IEEEkeywords}
embedded system, game, OLED
\end{IEEEkeywords}

\section{Introduction}
The CC3200 Launch Pad and OLED used for UC Davis EEC 172 instruction can be used
 to implement simple games. The accelerometer built in to CC3200 can be used
 as main input, the OLED can be used as main display of the game. The two push
 buttons in CC3200 can serve as game control (e.g. pause, restart). The CC3200
 also provides WiFi capability, which allows us to do basic communication with
 the IoT services for score reporting etc.

\section{Motivation}
UC Davis EEC 172 labs requires implementing a simple game using OLED and
 accelerometer. However, students cannot easily see how a real-world embedded
 system game will face challenges on performance and memory limitations. Thus,
 I decide to implement a much complex game in order to reveal these
 difficulties. I will also incorporate knowledge from linear algebra, numerical
 methods, and computer vision to optimize the game. I will also add features
 like screen rotation and score fetching in this project.

\section{Approach}
This project uses the code from EEC 172 labs as templates. In the beginning of
 this project there are two goals in parallel. The first is to have a proof
 of concept work showing that all components used are functional and there are
 no pin conflicts when using all of them together. The second is to develop
 the logic for the game, which involves mainly software engineering. 

After a naive implementation of the game, I can see performance problems of the
 game. For example, An analysis in Section \ref{graphic-lib-analysis} shows that
 the provided graphics library contains severe inefficiencies when drawing
 characters in the OLED. So optimization on the code should be performed. For
 example only re-draw updates to a frame and caching the OLED state.

Finally, some additional features like screen rotation and score fetching can
 be added.

\section{Components Used}
All components used in this project are provided in EEC 172 Lab. Specifically,
\begin{itemize}
\item CC3200 Launch Pad \cite{cc3200ug}
	\begin{itemize}
	\item Accelerometer \cite{bma222}
	\item WiFi module
	\item UART port (optional)
	\item Push buttons
	\item Slow Clock \cite{cc3200trm}
	\end{itemize}
\item OEL Display Module (OLED) \cite{oled}
\end{itemize}

A code similar to lab 4 is developed (not submitted) to prove that all the
 components can work independently without interfering each other.
 This becomes the ``proof of concept'' part of this project.

\subsection{Communication Protocols}
The CC3200 CPU communicates with the Accelerometer using I2C Protocol.

The CC3200 CPU communicates with the OLED using the SPI protocol.

\subsection{Pin Mux Configuration}
Pin Mux configuration is done using the TI PinMux tool.

\begin{itemize}
\item GPIO
	\begin{itemize}
	\item GPIO09: Pin 64, output, for WiFi LED
	\item GPIO13: Pin 4, input, for SW3
	\item GPIO22: Pin 15, input, for SW2
	\item GPIO25: Pin 21, output, for OLED R
	\item GPIO28: Pin 18, output, for OLED OC
	\item GPIO31: Pin 45, output, for OLED DC
	\end{itemize}
\item $\text{I}^\text{2}\text{C}$
	\begin{itemize}
	\item SCL: Pin 1, for accelerometer
	\item SDA: Pin 2, for accelerometer
	\end{itemize}
\item SPI
	\begin{itemize}
	\item CLK: Pin 5, for OLED CL
	\item MOSI: Pin 7, for OLED SI
	\end{itemize}
\item UART
	\begin{itemize}
	\item RX: Pin 57, for UART
	\item TX: Pin 55, for UART
	\end{itemize}
\end{itemize}

\subsection{Jumper Wire Connections}
Extra connections need to be made between the CC3200 Launch Pad and the OLED.
 7 Jumper wires are required in total. See Table \ref{tab:jwc} for details.

\begin{table}[htbp]
\caption{Jumper Wire Connections}
\begin{center}
\begin{tabular}{ | c | c | c | c | }\hline
\multicolumn{3}{|c|}{\textbf{CC3200}} & \textbf{OLED} \\ \hline
\textbf{Pin} & \textbf{Location} & \textbf{Ref} & \textbf{Pin} \\ \hline
VCC & J20 &  & + \\ \hline
GND & J20 &  & G \\ \hline
P05 & P1 & 7 & CL \\ \hline
P18 & P2 & 2 & OC \\ \hline
P45 & P2 & 4 & DC \\ \hline
P07 & P2 & 6 & SI \\ \hline
P21 & P2 & 8 & R \\ \hline
\end{tabular}
\label{tab:jwc}
\end{center}
\end{table}

\section{Game Logic}
The game main logic is implemented in the \lstinline{project_main()} function.
 It fetches the maximum score from Amazon AWS Shadow, and then call the
 \lstinline{game()} function. After game ends it sends the score and displays
 texts such as ``Game Over''.

The \lstinline{game()} function implements the game. It first initializes all
 variables such as the location and velocity of the ball and the tray. 
 The bricks are initialized in two ways: pre-defined pattern or random blocks.
 It also draws the OLED screen. Then it enters a while loop that iterates once
 per frame.

\begin{enumerate}
\item Compute the new location of the ball using Euler method in
	numerical methods, and compute collision between the board and the tray,
	the bricks, and the border. This part of the code extensively uses simple
	linear algebra computations for 2D vectors. The vectors are defined as a C
	struct called \lstinline{vect_t} and helper functions like
	\lstinline{normalize} and \lstinline{dot_prod} are written.
\item Draw the updated OLED screen, which overwrites the original
	position of the ball, the tray, and the broken bricks, and draws the new
	positions and the texts (e.g. score).
\item Check input (accelerometer reading, push button status).
\item Wait until the correct timing for next frame by polling the Slow Clock.
\end{enumerate}

\section{Optimizations}

\subsection{Analysis on Provided Graphics Library}
\label{graphic-lib-analysis}

An analysis to the provided graphics library in EEC 172 Lab 2 shows that there
 are a number of inefficiencies. One example is the \lstinline{drawChar()}
 function. A character of size 1 occupies $48$ pixels, and to draw one pixel
 \lstinline{drawChar()} calls \lstinline{fillRect()} of size 1 $48$ times.
 Each time \lstinline{fillRect()} need to transfer $9$ bytes to the OLED
 ($7$ bytes for setting location, $2$ bytes for the color of the rectangle).
 In total, $9 \times 48 = 432$ bytes need to be transferred for each character.

However, if we just use 7 bytes to set the location and then use
 $2 \times 48 = 96$ bytes to specify the pattern of the $8 \times 6$ rectangle,
 only $7 + 96 = 103$ bytes need to be transferred, saving more than $75\%$ of
 the time. 

\subsection{Rewriting Graphics Library}
I rewrote the graphics library provided. The only thing I reused is the
 \lstinline{glcdfont.h}, which defines the fonts. Since most objects in the 
 game are square-shaped, drawing them is straight forward. The characters are
 drawn using the method discussed in Section \ref{graphic-lib-analysis}. The
 ball is drawn by calculating the euclidean distance from the pixel to the
 center of the ball.

\subsection{Caching OLED memory}
The OLED is $128 \times 128$, and each pixel is represented by $2$ bytes.
 So to store the entire OLED memory, only $128 \times 128 \times 2 = 32768$
 bytes are needed, which is $32$ KiB, which is small enough to be stored
 in the memory.

The normal way the screen is initialized is: draw the black background, draw
 each brick, draw the ball, etc. This creates inefficiencies because the same
 pixel may be drawn multiple times.

Notice that the multiple writes are going through the SPI protocol, which is
 the bottleneck for performance. So we can cache the OLED state. Specifically,
 we compute the new OLED screen in memory (i.e. cache), and then flush it to
 OLED once the computation is done. We can also optimize by only flushing the
 part that are updated.

In the implementation, the OLED memory is cached as a global variable
 \lstinline{virt_oled} (in the data section). The \lstinline{render()} function
 flushes the data to OLED memory. The programmer can specify a rectangular
 region for \lstinline{render()} so that only the changed part of the screen can
 be updated.

\section{Extra Features}

\subsection{Score Fetching}
The game communicates with Amazon AWS service for storing and loading score.

It uses HTTP GET method to get the maximum score from AWS each time the
 CC3200 Launch Pad starts. There is a simple parser implemented using
 \lstinline{strstr()} and \lstinline{sscanf()} to extract the maximum score from
 the HTTP response. 

It uses HTTP POST method to send the current score and updated maximum score to
 AWS, and the settings in AWS can allow subscribers to receive text messages
 for score updates.

As a result, the user can see the maximum score in the board.

\subsection{Screen Rotation}
Using the spirit of the adapter method, caching OLED makes the communication
 with the OLED only through the \lstinline{render()} function. Thus, we can do
 simple tweaks to this function in order to support the 4 screen attitudes.

At the start of the game, the accelerometer is read and which attitude to use
 is computed.

We also need to do some changes to the logic converting the accelerometer
 reading to the tray speed.

\section{Reproducing}
Here are the procedures to reproduce this work

\subsection{Requirements}
\begin{itemize}
\item CC3200 Launch Pad (CC3200-LAUNCHXL)
	\begin{itemize}
	\item USB cable
	\end{itemize}
\item Adafruit 1.5'' SSD1351 128x128 RGB OLED
\item 7 jumper wires
\item A WiFi access point
\item An Amazon AWS account
\item Software such as Code Composer Studio, CCS UniFlash
\end{itemize}

\subsection{Wire Connections}
\begin{itemize}
\item Connect the OLED screen with the CC3200 Launch Pad. See Table \ref{tab:jwc}.
\item Connect the CC3200 to a computer using the USB cable.
\end{itemize}

\subsection{Setting Up AWS}
\label{aws}
\begin{enumerate}
\item Go to AWS IoT \cite{aws}, create a Thing. Create certificates for it and
	activate it. Also download Amazon AWS's Root CA.
\item Copy the REST API Endpoint host name from the Interact tab of the thing.
\item Go to IoT Policies and add a policy for allowing
	\lstinline{iot:GetThingShadow} and \lstinline{iot:UpdateThingShadow}.
\item Attach this policy to the certificates you have generated.
\item Use OpenSSL to convert the certificates to der format.
\item Create an SNS topic and create a subscription for it (with your phone
	number).
\item Create an IoT Rule that sends a message as an SNS push notification when
	the shadow updates.
\end{enumerate}

\subsection{Updating Code}
In \lstinline{main.c}, Update macro definitions for \lstinline{DATE},
 \lstinline{MONTH}, \lstinline{YEAR}, etc. using current date and time.
 Update \lstinline{SERVER_NAME}, \lstinline{POSTHEADER}, \lstinline{GETHEADER},
 \lstinline{HOSTHEADER} using AWS settings in Section \ref{aws}.

You also need to update your WiFi access point information in
 \lstinline{common.h} in the CC3200 SDK.

\subsection{Flashing Program}
The flashing configuration is saved in \lstinline{project.usf}.
 You need to change \lstinline{/sys/mcuing.bin} to use the project binary,
 \lstinline{/cert/rootCA.der}, \lstinline{/cert/client.der},
 \lstinline{/cert/private.der} to use the files from Section \ref{aws}.
 After you are done click the ``Program'' button.

\section{Future Work}

\begin{itemize}
\item Add more features to the game, like animation of brick disappearing.
\item Restructure the code to allow it easily be ported to other platforms
	(e.g. computers).
\item Allow users to select game difficulty.
\item Allow multiple users to exchange their game scores in AWS (currently
	if multiple players are playing the game there will be a race condition
	on score updating).
\end{itemize}

\section{Conclusion}
This project implements a realistic brick breaker game from scratch using the
 CC3200 Launch Pad and OLED provided in UC Davis EEC 172 class. The game makes
 use of the accelerometer connected through $\text{I}^\text{2}\text{C}$, the
 OLED connected through SPI, the WiFi module, the UART port, and Slow Clock. It
 contains game logic and collision detection algorithms, and includes highly
 optimized code for graphics operations. It also have features like rotating
 screen and fetching scores.

\section*{Acknowledgment}
I would like to thank the UC Davis Electrical and Computer Engineering
 Department for providing equipments (CC3200 Launch Pad, OLED) for this project.
 I would also like to thank Professor Homayoun and EEC 172 TAs for their
 instructions.

\begin{thebibliography}{00}
\bibitem{cc3200ug} Texas Instruments, ``CC3200 SimpleLink Wi-Fi and Internet of Things Solution With MCU LaunchPad Hardware User's Guide,'' June 2014, Revised March 2020. % https://www.ti.com/lit/ug/swru372c/swru372c.pdf
\bibitem{bma222} Bosch Sensortec, ``BMA 222 Digital, triaxial acceleration sensor Data sheet,'' May 2012. % https://media.digikey.com/pdf/Data%20Sheets/Bosch/BMA222.pdf
\bibitem{cc3200trm} Texas Instruments, ``CC3200 SimpleLink Wi-Fi and Internet-of-Things Solution, a Single Chip Wireless MCU Technical Reference Manual,'' June 2014, Revised May 2018. % https://www.ti.com/lit/ug/swru367d/swru367d.pdf
\bibitem{oled} Univision Technology Inc, ``OEL Display Module Product Specification,'' October 2008. % https://datasheetspdf.com/pdf/670326/UnivisionTechnology/UG-2828GDEDF11/1
\bibitem{aws} Amazon Web Services, ``AWS IoT Developer Guide,'' April 2020.
\end{thebibliography}

\end{document}
